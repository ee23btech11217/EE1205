%\iffalse
\let\negmedspace\undefined
\let\negthickspace\undefined
\documentclass[journal,12pt,twocolumn]{IEEEtran}
\usepackage{cite}
\usepackage{amsmath,amssymb,amsfonts,amsthm}
\usepackage{algorithmic}
\usepackage{graphicx}
\usepackage{textcomp}
\usepackage{xcolor}
\usepackage{txfonts}
\usepackage{listings}
\usepackage{enumitem}
\usepackage{mathtools}
\usepackage{gensymb}
\usepackage{comment}
\usepackage[breaklinks=true]{hyperref}
\usepackage{tkz-euclide} 
\usepackage{listings}
\usepackage{gvv}      
\usepackage{tikz}
\def\inputGnumericTable{}                                 
\usepackage[latin1]{inputenc}                                
\usepackage{color}                                            
\usepackage{array}                                            
\usepackage{longtable}                                       
\usepackage{calc}                                             
\usepackage{multirow}                                         
\usepackage{hhline}                                           
\usepackage{ifthen}                                           
\usepackage{lscape}
\usepackage[export]{adjustbox}

\newtheorem{theorem}{Theorem}[section]
\newtheorem{problem}{Problem}
\newtheorem{proposition}{Proposition}[section]
\newtheorem{lemma}{Lemma}[section]
\newtheorem{corollary}[theorem]{Corollary}
\newtheorem{example}{Example}[section]
\newtheorem{definition}[problem]{Definition}
\newcommand{\BEQA}{\begin{eqnarray}}
\newcommand{\EEQA}{\end{eqnarray}}
\newcommand{\define}{\stackrel{\triangle}{=}}
\theoremstyle{remark}
\newtheorem{rem}{Remark}

\begin{document}
\parindent 0px
\bibliographystyle{IEEEtran}

\vspace{3cm}

\title{}
\author{EE23BTECH11217 - Prajwal M$^{*}$
}
\maketitle
\newpage
\bigskip

% \renewcommand{\thefigure}{\theenumi}
% \renewcommand{\thetable}{\theenumi}


\section*{Exercise 9.1}

\noindent \textbf{12} \hspace{2pt}For the block diagram shown in the figure, the transfer function $\frac{Y\brak{s}}{R\brak{s}}$ is \\

\tikzset{
    block/.style = {draw, fill=white, rectangle, minimum height=1cm, minimum width=1cm},
    plus/.style= {draw, fill=white, circle, node distance=1cm, append after command={\pgfextra \draw ($(\tikzlastnode.center) + (-0.15,0)$) -- ($(\tikzlastnode.center) + (0.15,0)$) node[above] {$+$}; \endpgfextra}},
    input/.style = {coordinate},
    output/.style = {coordinate}
}

\begin{tikzpicture}[node distance=2cm,>=latex]
    \node [input] (input) at (0,0){};
    \node [block] (block1) at (2,-2){$2$};
    \node [block] (block2) at (6,-2){$3$};
    \node [plus] (sum1) at (2,-4) {};
    \node [plus] (sum2) at (6,-4){};
    \node [block] (block3) at (4,-4) {$\frac{1}{s}$};
    \node [output] (output) at (8,-4){};
    \node [block] (block4) at (4,-6){1};
    
    \draw (input) -- node[above]{$R\brak{s}$} (2,0) to (6,0);
    \draw [->] (6,0) -- (block2);
    \draw [->] (2,0) -- (block1);
    \draw [->] (block1) -- (sum1);
    \draw [->] (sum1) -- (block3);
    \draw [->] (block3) -- (sum2);
    \draw [->] (sum2) -- node[above]{$Y\brak{s}$}(output);
    \draw [->] (block2) -- (sum2);
    \draw (7,-4) to (7,-6);
    \draw [->] (7,-6) -- (block4);
    \draw (block4) to (2,-6);
    \draw [->] (2,-6) to (sum1);  
\end{tikzpicture}


\end{document}
