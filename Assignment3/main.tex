% \ffalse
\documentclass[journal,12pt,twocolumn]{IEEEtran}
\usepackage{amsmath,amssymb,amsfonts,amsthm}
\usepackage{txfonts}
\usepackage{tkz-euclide} 
\usepackage{listings}
\usepackage{gvv}       
\usepackage[latin1]{inputenc}   
\usepackage{array}  
\usepackage{tikz}
\usepackage{circuitikz}

\begin{document}

\bibliographystyle{IEEEtran}

\vspace{3cm}

\title{}
\author{EE23BTECH11217 - Prajwal M$^{*}$
}
\maketitle
\newpage
\bigskip

\renewcommand{\thefigure}{\theenumi}
\renewcommand{\thetable}{\theenumi}

\section*{Exercise 9.1}
The given figure shows a series LCR circuit connected to a variable
frequency 230 V source. \\
L = 5.0 H, C = 80 $\mu$F, R = 40 $\Omega$.

\begin{figure}[h!]
\begin{center}
\begin{circuitikz}[american voltages]
      \draw (0,0)
      to[sV, l=$\varepsilon$] (0,2) 
      to[R, l=$R$, v=$v_R$] (4,2) 
      to[C, l=$C$, v=$v_C$] (4,0)
      to[L, l=$L$, v=$v_L$] (0,0);
\end{circuitikz}
\end{center}
\end{figure}

\begin{enumerate}
    \item Determine the source frequency which drives the circuit in resonance.
    \item Obtain the impedance of the circuit and the amplitude of current
at the resonating frequency.
    \item Determine the rms potential drops across the three elements of
the circuit. Show that the potential drop across the LC
combination is zero at the resonating frequency.\\
\end{enumerate}

Solution:
\begin{table}[h]
    \centering
    \begin{tabular}{|c|c|c|}
\hline
\textbf{Paramater} & \textbf{Description} & \textbf{Value}  \\ \hline
$V\brak{t}$ & Voltage power supply & $230\sqrt{2} cos\brak{2\pi f t}$ V  \\ \hline
$V\brak{s}$ & Laplace transform of $V\brak{t}$ & ? \\\hline
L & Inductance & $5.0$ H  \\ \hline
C & Capacitance & $80\,\mu$F \\ \hline
R & Resistance & $40\,\Omega$ \\ \hline
$f$ & Frequency of voltage source & ? \\ \hline 
$Z$ & Impedance of circuit & ? \\ \hline
$V_R\brak{t}$ & Potential drop across Resistor & ?\\ \hline
$V_R\brak{s}$ & Laplace transform of $V_R\brak{s}$ & ?\\\hline
$V_C\brak{t}$ & Potential drop across Capacitor & ?\\ \hline
$V_C\brak{s}$ & Laplace transform of $V_C\brak{s}$ & ?\\\hline
$V_L\brak{t}$ & Potential drop across Inductor & ?\\ \hline
$V_L\brak{s}$ & Laplace transform of $V_L\brak{s}$ & ?\\\hline
\end{tabular}

    \caption{Parameter description}
    \label{tab:my_label}
\end{table}
\begin{figure}[h!]
\begin{center}
\begin{circuitikz}[american voltages]
      \draw (0,0)
      to[sV, l=$\varepsilon$] (0,2) 
      to[R, l=$R$, v=$V_R$] (4,2) 
      to[C, l=$\frac{-1}{sC}$, v=$V_C$] (4,0)
      to[L, l=$sL$, v=$V_L$] (0,0);
\end{circuitikz}
\caption{s-domain circuit diagram}
\label{fig: s-domain circuit}
\end{center}
\end{figure}
\begin{enumerate}
\item 
\begin{align}
    V_L + V_C & = 0 \\
    s_{\text{res}}L - \frac{1}{s_{\text{res}}C} & = 0\\
    s_{\text{res}} & = \frac{1}{\sqrt{LC}} \\
    s_{\text{res}} & = \frac{1}{\sqrt{(5.0 \, \text{H})(80 \times 10^{-6} \, \text{F})}} \\ 
    % \omega_{\text{res}} & = \sqrt{\frac{1}{4 \times 10^{-4}}} \\
    % \omega_{\text{res}} & \approx \sqrt{2500} \\
    s_{\text{res}} & = 50 \, \text{rad/s} 
\end{align}

% The source frequency which drives the circuit in
% resonance is 50 rad/s. \\
\item 
\begin{align}
    % Z & = R + j(\omega L - \frac{j}{\omega C})\\
    % \notag\text{at resonance ($\omega = \omega_{res}$),}\\
    Z_{\text{res}} & = R = 40 \,\Omega\\
    I_{\text{res}} & = \frac{\sqrt{2}\varepsilon}{Z_{\text{res}}} \\
    & = \frac{\sqrt{2}(230)}{40} \\
    & = 8.1 \, \text{A} 
\end{align}

% The amplitude of current (\(I_{\text{res}}\)) at resonance,

% Therefore, at the resonating frequency, the impedance of the circuit is \(40 \, \Omega\) and the amplitude of the current is \(8.1 \, \text{A}\).\\

\item 
% \(\omega_{\text{res}}\) = \(50 \, \text{rad/s}\), \(I_{\text{res}}\) is \(8.1 \, \text{A}\) (as calculated earlier).

\begin{align}
    I_{\text{rms}} & = \frac{I_{\text{res}}}{\sqrt{2}} \\
    & = \frac{8.1}{\sqrt{2}} \\
    & \approx 5.75 \, \text{A}\\
    V_R & = I_{\text{rms}}   R \\
    & = 5.75 \, \text{A} \times 40 \, \Omega \\
    & \approx 230 \, \text{V} \\
    V_L & = I_{\text{rms}}   s_{\text{res}} L \\
    & = 5.75 \, \text{A} \times 50 \, \text{rad/s} \times 5.0 \, \text{H} \\
    & \approx 1437.5 \, \text{V}  \\
    V_C & = I_{\text{rms}}   \frac{-1}{s_{\text{res}} C} \\
    & = 5.75 \, \text{A} \times \frac{-1}{50 \, \text{rad/s} \times 80 \times 10^{-6} \, \text{F}} \\
    & \approx -1437.5 \, \text{V} \\
    V_{LC} & = I_{\text{rms}}   s_{\text{res}} L - I_{\text{rms}}   \frac{1}{s_{\text{res}} C} \\
    & = V_L + V_C \\
    & = 1437.5 \, \text{V} - 1437.5 \, \text{V} = 0 \, \text{V}
\end{align}
\end{enumerate}
\end{document}
i
